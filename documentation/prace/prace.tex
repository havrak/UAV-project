\documentclass[a4paper,oneside,12pt]{report}
\setlength\textwidth{155mm}
\setlength\textheight{247mm}
\setlength\topmargin{-10mm}
\setlength\headheight{0mm}
\setlength\oddsidemargin{05mm}
\setlength\evensidemargin{05mm}
\let\openright=\clearpage


%% Vytváříme PDF/A-2u
\usepackage[a-2u]{pdfx}

%% Přepneme na českou sazbu a fonty Latin Modern
\usepackage[czech]{babel}
\usepackage[utf8]{inputenc}
\usepackage{lmodern}
\usepackage[T1]{fontenc}
\usepackage{textcomp}


\usepackage{amsmath}						% rozšíření pro sazbu matematiky
\usepackage{amsfonts}						% matematické fonty
\usepackage{amsthm}							% sazba vět, definic apod.
\usepackage{bbding}							% balíček s nejrůznějšími symboly (čtverečky, hvězdičky, tužtičky, nůžtičky, ...)
\usepackage{bm}									% tučné symboly (příkaz \bm)
\usepackage{graphicx}						% vkládání obrázků
\usepackage{fancyhdr}						% možnost stylizovat záhlaví
\usepackage{fancyvrb}						% vylepšené prostředí pro strojové písmo
\usepackage{indentfirst}				% zavede odsazení 1. odstavce kapitoly
\usepackage[nottoc]{tocbibind} 	% zajistí přidání seznamu literatury,
\usepackage{icomma}         		% inteligentní čárka v matematickém módu
\usepackage{dcolumn}        		% lepší zarovnání sloupců v tabulkách
\usepackage{booktabs}       		% lepší vodorovné linky v tabulkách
\usepackage{paralist}       		% lepší enumerate a itemize
\usepackage{caption}						%	popisky
\usepackage{dirtree}						% strom souborů
\usepackage{minted} 						% vkládání kódu
\usepackage[bottom]{footmisc}  	% poznámky pod čarou vespod
\usepackage{bibentry} 					% citace přes celé heslo, používáno u první citace z jednoho zdroje
\usepackage{xurl}								% umožní url zalomit všude
\usepackage{hyperref} 					% odstranění červených okrajů v obsahu
\nobibliography*

\usepackage{color}
\usepackage{natbib}

\definecolor{pblue}{rgb}{0.13,0.13,1}
\definecolor{pgreen}{rgb}{0,0.5,0}
\definecolor{pred}{rgb}{0.9,0,0}
\definecolor{pgrey}{rgb}{0.46,0.45,0.48}
\renewcommand{\baselinestretch}{1.5}
%%% Údaje o práci

\def\NazevSkoly{Gymnázium, Praha 6, Arabská 14}
% Název oboru včetně počátečního 'Obor'.
\def\NazevOboru{Programování}

% Název práce v jazyce práce (přesně podle zadání)
\def\NazevPrace{Bezpilotní letadlo}

\def\NazevPraceShort{Bezpilotní letadlo}
% Název práce v angličtině
\def\NazevPraceEN{Drone}

% Název práce v němčině
\def\NazevPraceDE{Drohne}

% Jméno autora
\def\AutorPrace{Havránek Kryštof}

% Rok odevzdání
\def\RokOdevzdani{2022}
% Měsíc odevzdání
\def\MesicOdevzdani{Únor}

% Vedoucí práce: Jméno a příjmení s~tituly
\def\Vedouci{ing. Daniel Kahoun}

% Nepovinné poděkování (vedoucímu práce, konzultantovi, tomu, kdo
% zapůjčil software, literaturu apod.)
\def\Podekovani{%
\textbf{Poděkování}
}

% Abstrakt (doporučený rozsah cca 80-200 slov; nejedná se o zadání práce)
\def\Abstrakt{%
}

\def\AbstraktEN{%
}
\def\AbstraktDE{%
}
% 3 až 5 klíčových slov (doporučeno), každé uzavřeno ve složených závorkách
\def\KlicovaSlova{%
	(UAV), (Gtk), (C++), (Raspberry zero 2)
}
\def\KlicovaSlovaEN{%
	(UAV), (Gtk), (C++), (Raspberry zero 2)
}
\def\KlicovaSlovaDE{%
	(UAV), (Gtk), (C++), (Raspberry zero 2)
}

%% Balíček hyperref, kterým jdou vyrábět klikací odkazy v PDF,
%% ale hlavně ho používáme k uložení metadat do PDF (včetně obsahu).
\hypersetup{unicode}
\hypersetup{breaklinks=true}
\hypersetup{pdfborder=0 0 0}

%% Definice různých užitečných maker (viz popis uvnitř souboru)
\include{makra}

%% Titulní strana a různé povinné informační strany
\fancypagestyle{plain}{
\fancyhf{}
\renewcommand{\headrulewidth}{0.4pt}
\renewcommand{\footrulewidth}{0.4pt}
\fancyhead[C]{}
\fancyhead[L]{Ročníková práce -- \NazevSkoly}
\fancyhead[R]{\NazevPraceShort}
\fancyfoot[L]{Vypracoval: \AutorPrace	(\NazevOboru)}
\fancyfoot[C]{}
\fancyfoot[R]{\thepage}
}



\begin{document}

%%% Titulní strana práce

\pagestyle{empty}
\hypersetup{pageanchor=false}

\begin{center}

{\LARGE\bfseries\NazevSkoly}

\vspace{-18mm}
\vfill

{\LARGE\NazevOboru}

\vfill

\centerline{\mbox{\def\svgwidth{\columnwidth}\scalebox{0.25}{\input{../img/logo.pdf_tex}}}}

\vspace{-8mm}
\vfill

{\bf\Large ROČNÍKOVÝ PROJEKT}

\vfill


\vspace{15mm}

{\LARGE\bfseries\NazevPrace}


\vfill


Vypracoval: \hfill \AutorPrace

Vedoucí práce: \hfill \Vedouci

\vspace{15mm}
\MesicOdevzdani \ \RokOdevzdani

\end{center}



\newpage
\hypersetup{pageanchor=true}
\pagestyle{plain}
\pagenumbering{roman}


\openright


\vspace*{\fill}


\noindent
Prohlašuji, že jsem jediným autorem tohoto projektu, všechny citace jsou
řádně označené a všechna použitá literatura a další zdroje jsou v práci uvedené.
Tímto dle zákona 121/2000 Sb. (tzv. Autorský zákon) ve znění pozdějších předpisů uděluji
bezúplatně škole Gymnázium, Praha 6, Arabská 14 oprávnění k výkonu práva na rozmnožování díla
(§ 13) a práva na sdělování díla veřejnosti (§ 18) na dobu časově neomezenou a bez omezení
územního rozsahu.


\vspace{1cm}

\noindent
V ........ dne ............
\hspace{4cm}
Podpis autora


\newpage


\openright

\vbox to 0.20\vsize{
\setlength\parindent{0mm}
\setlength\parskip{5mm}

Název práce:
\NazevPrace

Autor:
\AutorPrace

% Vedoucí práce:
% \Vedouci, \KatedraVedouciho

Abstrakt:
\Abstrakt

Klíčová slova:
\KlicovaSlova

% Opakování v angličtině.
\noindent\rule{7cm}{0.4pt}

Title:
\NazevPraceEN

Author:
\AutorPrace

Abstract:
\AbstraktEN

Key words:
\KlicovaSlovaEN


% Opakování v němčině.
\noindent\rule{7cm}{0.4pt}

Titlel:
\NazevPraceDE

Autor:
\AutorPrace

Abstrakt:
\AbstraktDE

Schlüsselwörter:
\KlicovaSlovaDE

\vss}

\newpage

\openright



\tableofcontents


\newpage

\chapter*{Úvod}
\addcontentsline{toc}{chapter}{Úvod}

Cílem práce bylo sestavit bezpilotní letadlo a napsat doprovodný software -- a to jak pro samotný dron, tak počítač, který letadlo ovládá.
Uživatel sytému by měl mít možnost sledovat video stream z letadla a letadlo dálkově ovládat.
Programová část kódu by měla být psána v souladu s klasickými návrhovými vzory a umožňovat jednoduchou implementaci dalších funkcí.

Dron je postavený okolo malého počítače Raspberry Pi Zero 2, ke kterému jsou připojeny ostatní periférie přes sériovou a I2C linku.
Komunikace mezi dronem a pilotem probíhá přes TCP protokol.
Raspberry Pi tak funguje jako Access Point.
Konfigurační soubory pro nastavení Access Pointu jsou přiložené u kódu.

Celý kód práce je dostupný na Github pod licencí MIT -- \url{https://github.com/havrak/UAV-project}.
V rámci práce na projektu byly napsány tři doprovodné knihovny -- \url{https://github.com/havrak/Raspberry-JY901-Serial-I2C}, \url{https://github.com/havrak/raspberry-pi-ina226} a \url{https://github.com/havrak/PCA9685-rpi}.
Všechny jsou volně dostupné k použití také pod licencí MIT.



\pagenumbering{arabic}
\setcounter{page}{1}

\chapter{Rozložení práce}

Práci lze rozdělit na dva základní komponenty.
První část software běží na počítači přes který se bezpilotní letadlo ovládá, druhá na samotném dronu.
Obě části jsou psané v jazyce C++.

Software pro počítač používá grafickou knihovnu Gtk3 a byl vyvíjen primárně na operačním systém Linux.
Gtk3 je multiplatformní toolkit, port na další operační systémy je tak možný a kód je psán stylem, aby umožnil další verze.

Jádro samotného dronu představuje malý počítač Raspberry Pi Zero 2.
Vyšší výkon verze Zero verze 2 není nutný pro fungování, práce tak defacto může fungovat na libovolném Raspberry Pi.

Obě části mezi sebou komunikují prostřednictvím protitoku založené na TCP rodině.
Přenos videa je zprostředkovávám pomocí UDP.

\section{Komunikační protokol}



\chapter{Hardware a Raspberry Pi}

\chapter{Client na ovládání}




\chapter*{}
\pagenumbering{roman}
\setcounter{page}{6} % text bude pokračovat v číslování římském, kde přestal na začátku souboru
\chapter*{Závěr}
\addcontentsline{toc}{chapter}{Závěr}

Cíl práce specifikovaný v zadání byl splněn.
Letadlo je schopné letu a uživatel má k dispozici stream z kamery na dronu.

Práci však komplikovala řada problémů a to jak s hardwarem, tak se softwarem.
Před začátkem jsem měl jen velmi povrchní zkušenosti s jazykem C++ a žádné s jak Gtk3, tak vývojem pro Raspberry Pi.
Práce tak nemohla být realizována tak, jak byla původně zamýšlena.

Tomu napovídá i dlouhý list specifikující budoucnost projektu.
Ačkoliv tomu samotné řádky kódu nemusí napovídat práce mi dala celou řadu nových znalostí.
I proto plánuji dále na projektu pokračovat ve svém volém čase.
Rád bych systém dostal do stavu kdy se jedná o plnohodnotnou realizaci bezpilotního systému.



\bibliographystyle{czplainnat}   %% Autor (rok) s českými spojkami
%\bibliographystyle{plainnat}    %% Autor (rok) s anglickými spojkami
%\bibliographystyle{unsrt}       %% [číslo]


\renewcommand{\bibname}{Seznam použité literatury}

\bibliography{literatura}

%%% Kdybyste chtěli bibliografii vytvářet ručně (bez bibTeXu), lze to udělat
%%% následovně. V takovém případě se řiďte normou ISO 690 a zvyklostmi v oboru.

% \begin{thebibliography}{99}
%
% \bibitem{lamport94}
%   {\sc Lamport,} Leslie.
%   \emph{\LaTeX: A Document Preparation System}.
%   2. vydání.
%   Massachusetts: Addison Wesley, 1994.
%   ISBN 0-201-52983-1.
%
% \end{thebibliography}



\listoffigures
\openright
\end{document}
