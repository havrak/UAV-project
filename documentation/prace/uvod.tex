\chapter*{Úvod}
\addcontentsline{toc}{chapter}{Úvod}

Cílem práce bylo sestavit bezpilotní letadlo a napsat doprovodný software -- a to jak pro samotný dron, tak počítač, který letadlo ovládá.
Uživatel sytému by měl mít možnost sledovat video stream z letadla a letadlo dálkově ovládat.
Programová část kódu by měla být psána v souladu s klasickými návrhovými vzory a umožňovat jednoduchou implementaci dalších funkcí.

Dron je postavený okolo malého počítače Raspberry Pi Zero 2, ke kterému jsou připojeny ostatní periférie přes sériovou a I2C linku.
Komunikaci mezi dronem a pilotem probíhá přes TCP protokol.
Raspberry Pi tak funguje jako Access Point, konfigurační soubory pro nastavení Access Pointu jsou součástí práce.

Celý kód práce je dostupný na github pod licencí MIT -- \url{https://github.com/havrak/UAV-project}.
V rámci práce na projektu byly napsány dvě doprovodné knihovny -- \url{https://github.com/havrak/Raspberry-JY901-Serial-I2C} a \url{https://github.com/havrak/raspberry-pi-ina226}.
Obě jsou volně dostupné k použití také pod licencí MIT.

